\documentclass[11pt, letterpaper, twocolumn, fleqn]{article}
\usepackage[margin=0.5in]{geometry}
\usepackage[utf8]{inputenc}
\usepackage{amsmath,amssymb,amsthm,graphicx}
\graphicspath{ {./images/} }

\let\oldemptyset\emptyset
\let\emptyset\varnothing

\begin{document}
\renewcommand{\labelenumi}{\alph{enumi}.}
\renewcommand{\labelenumii}{(\arabic{enumii})}
\renewcommand{\qedsymbol}{$\blacksquare$}

\paragraph{5.1.1}
\begin{enumerate}
  \item $(9 + 26)^7 = 35^7$
  \item $9 \cdot 35^7$
\end{enumerate}

\paragraph{5.1.2}
\begin{enumerate}
  \item $(4 + 10 + 26)^6 = 40^6$
  \item $40^7 + 40^8 + 40^9$
\end{enumerate}

\paragraph{5.1.3}
\begin{enumerate}
  \item $10^5$
  \item $10^4 \cdot 3$
  \item $10^3 \cdot 3^2$
  \item $10^5 \cdot 2$
\end{enumerate}

\paragraph{5.2.2}
\begin{enumerate}
  \item Let $f(x)$ be a function where $x \in B^n$ and $f(x)$ appends $x$ to $x$. For example, if $x = 101$ then $f(101) = 101101$. The inverse of this function would be to remove the last 3 characters. For example, $f^{-1}(101101) = 101$. Since $f^{-1}(f(x)) = x$, $f(x)$ is a bijection.
  \item
  \item
\end{enumerate}

\paragraph{5.2.3}
\begin{enumerate}
  \item 
  \item
\end{enumerate}

\end{document}
