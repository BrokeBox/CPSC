\documentclass[11pt, letterpaper, twocolumn, fleqn]{article}
\usepackage[margin=0.5in]{geometry}
\usepackage[utf8]{inputenc}
\usepackage{amsmath,amssymb,amsthm,graphicx, textcomp, siunitx}

\graphicspath{ {./images/} }

\let\oldemptyset\emptyset
\let\emptyset\varnothing

\sisetup{output-exponent-marker=\ensuremath{\mathrm{e}}}

\begin{document}
\renewcommand{\labelenumi}{\alph{enumi}.}
\renewcommand{\labelenumii}{(\arabic{enumii})}
\renewcommand{\qedsymbol}{$\blacksquare$}

\widowpenalties 1 10000
\raggedbottom
\pagestyle{headings}

\paragraph{6.1.1}
\begin{enumerate}
  \item \{HTTH, HHTH, HTHH, HHHH\}
  \item \{HHHH, HHHT, HHTH, HTHH, THHH, HHTT, THHT, TTHH\}
  \item \{THHH, TTHH, THHT\}
\end{enumerate}

\paragraph{6.1.2}
\begin{enumerate}
  \item The number of ways the kids can line up is $n!$. If Celia is first, we pick the first position and finish picking the kids in line. So the number of ways Celia can come first is $1 \cdot (n-1)!$. The probability that Celia comes first is 
    $$\frac{(n-1)!}{n!}$$
  
  \item If Celia is first and Felicity is last, then two positions out of $n$ are fixed. Therefore the cardinality of the event is $|E|=(n-2)!$. Then the probability becomes
    $$\frac{(n-2)!}{n!}$$
    
  \item First we decide if Celia is to the left or right of Felicity, which gives 2 choices. Then we count the combinations of $n-2$ kids and the Felicity/Celia pair: $2 \cdot (n-2)!$. The probability the two stand next to each other is:
    $$\frac{2(n-2)!}{n!}$$
\end{enumerate}

\paragraph{6.1.4}
\begin{enumerate}
  \item The size of the sample space is $\binom{10}{5} = 252$
  \item Place Alex and Jose on a team together and pick the rest of the kids for that team. Multiply by two to get the probability for team A and B.
    $$2 \cdot \binom{8}{3} / \binom{10}{5} = \frac{4}{9}$$
\end{enumerate}

\paragraph{6.1.5}
\begin{enumerate}
  \item The number of ways to pick a 5 card hand is $\binom{52}{5}$. There are 13 ways to select the rank of the first 3 cards. After the rank has been determined there are $\binom{4}{3}$ ways to select the 3 cards of that rank from the 4 suits. Then there are $13-1$ ways to select the rank for the pair, and $\binom{4}{2}$ ways to select the suit. All of this combined gives:
    $$\frac{13\binom{4}{3}(13-1)\binom{4}{2}}{\binom{52}{5}} = \frac{6}{4165}$$
    
  \item Picking the rank and suit for the three of a kind card gives $13 \cdot \binom{4}{3}$. There are 12 ranks remaining for the other two cards, picking two from the 12 is $\binom{12}{2}$. Once the ranks are chosen, ther are 4 ways to pick the suit for each. All together gives:
    $$\frac{13 \binom{4}{3} \binom{12}{2} \cdot 4 \cdot 4}{\binom{52}{5}} = \frac{88}{4165}$$
\end{enumerate}

\paragraph{6.2.1}
\begin{enumerate}
  \item The size of the sample space is $2^n$. If $n-1$ flips come up heads then there are only two possible choices, the outcome where the last flip is heads and the outcome where the last flip is tails. Adding these two probabilities together gives:
    $$\frac{1}{2^n} + \frac{1}{2^n} = \frac{2}{2^n} = 2^{1-n}$$
  
  \item There are two ways for there to be no consecutive flips. To start with heads and alternate flips $n$ times, or to start with tails and alternate flips $n$ times. Adding these together and taking the complement gives the probability that at least two consecutive flips are the same:
    $$1 - \left(\frac{1}{2^n}+\frac{1}{2^n}\right) = 1 - 2^{1-n}$$
    
  \item From $n$ possibilites, pick $n/2$ outcomes for heads, $\binom{n}{n/2}$. This is the number of ways that the number of heads is the same as the number of tails.  Taking the complement gives the number of ways the number of heads is different from the number of tails:
  \begin{align*}
    1-\frac{\binom{n}{n/2}}{2^n} &= 1- \frac{n!}{(n/2)!(n-n/2)!} \cdot \frac{1}{2^n}\\
    &= 1-\frac{n!}{(2^n)(\frac{n}{2}!)^2}
  \end{align*}
\end{enumerate}

\paragraph{6.2.2}
\begin{enumerate}
  \item The number of ways that the kids can line up is $n!$. The number of ways that a student is first is $(n-1)!$. The cardinality of the intersection of the Events where Celia is first and Felicity is first equals zero, i.e. $|C| \cap |F| = 0$. Therefore the probability that Celia or Felicity is first in line is:
    $$\frac{(n-1)!}{n!} + \frac{(n-1)!}{n!} = \frac{2(n-1)!}{n!}$$
    
  \item The number of ways that Celia is first is $(n-1)!$, the number of ways Felicity is last is $(n-1)!$. The intersection of the two events is not zero for the one case where Celia is first and Felicity is last. Therefore the probability that Celia is first and Felicity is last is:
    $$\frac{2(n-1)!}{n!} - 1$$
    
  \item The probability that Celia and Felicity are next to each other in line is $\frac{2(n-2)!}{n!}$ as shown in \S 6.1.2.c. Therefore the probability that they don't stand next to each other is 
    $$1 - \frac{2(n-2)!}{n!}$$
\end{enumerate}

\paragraph{6.2.5}
\begin{enumerate}
  \item The cardinality of the union of the 3 sets of characters is $26+26+10 = 62$. The cardinality of the sample space then is $62^{10}$. The cardinality of the events that contain only one of the character sets is $26^{10}$, $26^{10}$, and $10^{10}$. These events are all mutually exclusive.
  
  Then there are the events that contain two character sets. They are
  \begin{align*}
    |LETTERS \cup numbers| &= 36^{10}\\
    |letters \cup numbers| &= 36^{10}\\
    |letters \cup LETTERS| &= 52^{10} 
  \end{align*}
  
  Combining all of these together gives the probability that a randomly selected string is not a valid password. Since the events containing only one of the character sets are not disjoint from the events containg two, we subtract where appropriate. Therefore the number of valid password is:
  \begin{align*}
    2&\cdot(36^{10} - 26^{10} - 10^{10}) + (52^{10} - 26^{10} - 26^{10}) \\
    &+26^{10} + 26^{10} + 10^{10} = 1.52\cdot10^{17}
  \end{align*}
  
  And the probability that a valid password is chosen is:
    $$\frac{1.52\cdot10^{17}}{62^{10}} = .18$$
\end{enumerate}

\paragraph{6.3.1}
\begin{enumerate}
  \item The sum of two even numbers is even, the sum of an even and odd number is odd, and the sum of two odd numbers is even. The cardinality of the event that the sum on the two dice is even is $|\{1,3,5\}^2| + |\{2,4,6\}^2| = 18$. Then the probability is 
    $$p(A) = \frac{18}{36} = \frac{1}{2}$$
    
  The cardinality of the event that the sum of the two dice is at least time is $|\{(5,5),(5,6),(6,5),(6,6)\}| = 4$. Therefore the probability is:
    $$p(B) = \frac{4}{36} = \frac{1}{9}$$
    
  The probability that the red die comes up 5 is:
    $$p(C) = \frac{6}{36} = \frac{1}{6}$$
    
  \item A intersects C for the outcomes where the blue die is odd. Therefore $|A \cap C| = |\{(5,1),(5,3),(5,5)\}| = 3$ The probability of A given C is: 
    $$p(A|C) = \frac{|A \cap C|}{|C|} = \frac{3}{6} = \frac{1}{2}$$
    
  \item B intersects C for the outcomes where the blue die is 5 or 6. Therefore $|B \cap C| = |\{(5,5),(5,6)\}| = 2$. The probability of B given C is:
    $$p(B|C) = \frac{|B \cap C|}{|C|} = \frac{2}{6} = \frac{1}{3}$$
    
  \item A intersects B for the following outcomes: 
    $$\{(5,5),(6,6)\}$$
  Therefore $|A \cap B| = 2$. The probability of A given B is:
    $$p(A|B) = \frac{|A \cap B|}{|B|} = \frac{2}{4} = \frac{1}{2}$$
    
  \item $A$ and $B$ are independent because $p(A) = p(A|B)$. $A$ and $C$ are independent because $p(A) = p(A|C)$. $B$ and $C$ are not independent because $p(B) \neq p(B|C)$
\end{enumerate}

\paragraph{6.3.3}
\begin{enumerate}
  \item The bride can be to the left or the right of the groom. And now there are 7 items to order, 6 people and the bride-groom pair. This gives $2 \cdot 7!$ ways to arrange the party so the bride and groom are together. Divided by the total number of ways to arrange the party gives the probability that they will be next to each other:
    $$\frac{2\cdot 7!}{8!} = \frac{1}{4}$$
    
  \item Placing the maid of honor in the leftmost position gives $(8-1)!$ different ways to arrange the party. Therefore the probability is:
    $$\frac{7!}{8!} = \frac{1}{8}$$
    
  \item The intersection of the two events is when the maid of honor is in the left most position and the bride and groom are next to each other. First we place the maid of honor, and there are 7 positions left to choose. With 7 positions left there are $2 \cdot 6!$ ways to arrange the party so the bride and groom are together and the maid of honor is in the left most position. The probability of this event is:
    $$\frac{2 \cdot 6!}{8!} = \frac{1}{28}$$
    
  Because the product of the probabilities that the bride is next to the groom and the maid of honor is in the leftmost position does not equal the probability of their intersection, the events are not independent.
    $$\frac{1}{4} \cdot \frac{1}{8} \neq \frac{1}{28}$$
\end{enumerate}

\paragraph{6.3.6}
\begin{enumerate}
  \item $\left(\frac{1}{3}\right)^{10}$
  \item $\left(\frac{1}{3}\right)^{5} \cdot \left(\frac{2}{3}\right)^{5}$
\end{enumerate}

\paragraph{6.4.1}
\begin{enumerate}
  \item Let F be the event that Sally picked the biased coin. Then $p(F) = .5 = p(\overline{F})$. Let X be the event that 3 flips are heads. The probability that 3 flips are heads given a fair coin is $p(X|\overline{F}) = (.5)^{10}$. The probability that 3 flips are heads given a biased coin is $p(X|F) = (.25)^7 \cdot (.75)^3$. Then the probability that she selected a biased coin is:
  \begin{align*}
    p(F|X) &= \frac{p(X|F)p(F)}{p(X|F)p(F) + p(X|\overline{F})p(\overline{F})} \\
    &= \frac{(.25^7)(.75^3)(.5)}{(.25^7)(.75^3)(.5) + (.5^{10})(.5)} \\
    &= .026
  \end{align*}
\end{enumerate}

\paragraph{6.4.3}
\begin{enumerate}
  \item Let T be the event that a player tests positive and D be the event that a player takes drugs. We are trying to find the probability that a player is taking drugs, given they tested positive.
    $$p(D|T) = \frac{p(T|D)p(D)}{p(T|D)p(D) + p(T|\overline{D})p(\overline{D})} $$
  
  The probability that a player is taking drugs is $p(D) = .03$, and the probability that a player is not is $p(\overline{D}) = .97$. The false negative rate is $p(\overline{T}|D) = .04$. The true negative rate by complement is $p(T|D)= .96$. The false positive rate is $p(T|\overline{D}) = .02$. Using this to fill out the equation above gives a probability of
    $$p(D|T) = \frac{(.96)(.03)}{(.96)(.03)+(.02)(.97)} = .598 $$
  that the flufferball player is actually taking drugs.
\end{enumerate}

\paragraph{6.4.4}
\begin{enumerate}
  \item Let H be the event that a person has HIV and T be the event that a person tests positive for it. We are trying to find the probability that a person has HIV given they test positive.
    $$p(H|T) = \frac{p(T|H)p(H)}{p(T|H)p(H) + p(T|\overline{H})p(\overline{H})}$$
  
  Summary of probabilities
  \begin{itemize}
    \item $p(H) = .0001$ is the probability a person has HIV
    \item $p(\overline{H}) = .9999$ is the probability a person does not have HIV
    \item $p(T|\overline{H}) = .025$ is the probability that a person tests positive and doesnt have HIV
    \item $p(H|\overline{T}) = 0$ is the probability that you have HIV given a negative test
  \end{itemize} 
  
  If the test guarantees a negative means you do not have HIV then  $p(\overline{T}|H)$ must also equal 0 and by complement, $p(T|H) = 1$. Then the equation becomes
    $$p(H|T) = \frac{(1)(.0001)}{(1)(.0001)+(.025)(.9999)} = .004$$
\end{enumerate}

\paragraph{6.5.1}
\begin{enumerate}
  \item The range of $X$ is 
    $$\{1,2,3,4,5,6,8,9,10,12,15,$$
    $$16,18,20,24,25,30,36\}$$
  \item $p(X = 6) = \frac{|\{(2,3),(3,2),(1,6),(6,1)\}|}{36} = \frac{4}{36} = .111$
\end{enumerate}

\paragraph{6.5.2}
\begin{enumerate}
  \item The range of $A$ is $\{0,1,2,3,4\}$
  \item 
    Let $n \in A$, for each $n$ there are $\binom{4}{n}$ ways to select the suit of the ace(s) in a 5 card hand. The number of 5 card hands with $n$ aces is 
      $$\binom{4}{n} \binom{48}{5-n}$$
      
    The probability that there are $n$ Aces in a 5 card hand is 
      $$\binom{4}{n} \binom{48}{5-n} / \binom{52}{5}$$
    
    There fore the distribution over the random variable $A$ is
      $$\{(0,0.659),(1,0.299),(2,0.040),$$
      $$(3,\num{1.74e-3}),(4,\num{0.018e-3})\}$$
\end{enumerate}

\paragraph{6.5.5}
\begin{enumerate}
  \item The range of $X$ is:
    $$\{-10,-8,-6,-4,-2,$$
    $$0,2,4,6,8,10\}$$
  
  \item The cardinality of the sample space is $2^{10} = 1024$. Let $n \in A$, for each $n$ there are $\binom{10}{(10-n)/2}$ outcomes for the corresponding event in $A$. Therefore the distribution over $X$ is:
  
  \begin{center}
  \begin{tabular}{c c c}
     $\{\left(-10,\frac{1  }{1024} \right),$
      &$\left(-8 ,\frac{10 }{1024} \right),$ 
      &$\left(-6 ,\frac{45 }{1024} \right),$ \\
       $\left(-4 ,\frac{120}{1024} \right),$ 
      &$\left(-2 ,\frac{210}{1024} \right),$ 
      &$\left(0  ,\frac{252}{1024} \right),$ \\
       $\left(2  ,\frac{210}{1024} \right),$
      &$\left(4  ,\frac{120}{1024} \right),$
      &$\left(6  ,\frac{45 }{1024} \right),$ \\
       $\left(8  ,\frac{10 }{1024} \right),$
      &$\left(10 ,\frac{1  }{1024} \right) \}$ 
  \end{tabular}
  \end{center}
\end{enumerate}

\paragraph{6.6.1}
\begin{enumerate}
  \item The size of the sample space is $\binom{10}{2} = 45$. The number of girls chosen can be $r = \{0,1,2\}$. The distribution over $G$ is 
    $$\left\{
      \left(0,\frac{3 }{45}\right),
      \left(1,\frac{21}{45}\right),
      \left(2,\frac{21}{45}\right)
    \right\}$$
    
  The expectation for the number of girls chosen is 
    $$E[G] = 0\cdot\frac{3 }{45} + 1\cdot\frac{21}{45} + 2\cdot\frac{21}{45} = 1.4$$
\end{enumerate}

\paragraph{6.6.2}
\begin{enumerate}
  \item Let $X$ be the random variable denoting the amount the player wins or loses. The outcomes for $X$ are $\{-1,1,2\}$. The distribution over $X$ is 
    $$\left\{
      \left(-1,\frac{4}{6}\right),
      \left(1,\frac{1}{6}\right),
      \left(2 ,\frac{1}{6}\right)
    \right\}$$
    
  The expected amount that the player wins or loses is:
    $$E[X] = -1\cdot\frac{4}{6} + 1\cdot\frac{1}{6} + 2\cdot\frac{1}{6} = -\$0.17$$
    
  In other words, the player is expected to lose $\$0.17$.
\end{enumerate}

\paragraph{6.6.4}
\begin{enumerate}
  \item The outcomes for $X$ are $\{1,4,9,16,25,36\}$. The distribution over $X$ is 
    $$\bigg\{
      \left(1 ,\frac{1}{6}\right),
      \left(4 ,\frac{1}{6}\right),
      \left(9 ,\frac{1}{6}\right),$$
    $$\left(16,\frac{1}{6}\right),
      \left(25,\frac{1}{6}\right),
      \left(36,\frac{1}{6}\right)
    \bigg\}$$
    
  The expected result is 
    $$E[X] = (1 + 4 + 9 + 16 + 25 + 36)\frac{1}{6} = 15.2$$
\end{enumerate}

\paragraph{6.7.2}
\begin{enumerate}
  \item Let $F_i$ be a random variable that is 1 if the $i^{th}$ computer has the file and 0 if it does not. $F = F_1 + ... + F_7$. All $E[F_i]$ are the same. $p(F_1 = 1) = \frac{5}{40}$ because 5 out of 40 computers hold the file. Therefore $E[F_i] = 1\cdot\frac{5}{40} + 0 \cdot \frac{35}{40} = \frac{5}{40}$.
    $$E[F] = E[F_1] + ... + E[F_7] = 7 \left(\frac{5}{40}\right) = \frac{35}{40}$$
    
  Therefore, it is expected $7/8$ of the failed computers have the file.
\end{enumerate}

\paragraph{6.7.4}
\begin{enumerate}
  \item Let $C$ be the random variable that child gets their coat, and let $C_i$ be a random variable that is 1 if the $i^{th}$ child gets their coat and 0 if they do not. $C = C_1 + ... + C_{10}$. All $E[C_i]$ are the same. 
  
  $p(C_1 = 1) = 1/10$ because for the first student, there are 10 possible coats to get with only 1 being their own. Therefore $E[C_i] = 1 \cdot (1/10) + 0 \cdot (9 / 10) = 1/10$.
    $$E[C] = E[C_1] + ... + E[C_10] = 10 \cdot (1/10) = 1$$
    
  Therefore it is expected that $1$ student will get their own jacket.
\end{enumerate}

\paragraph{6.8.1}
\paragraph{6.8.3}
\end{document}
