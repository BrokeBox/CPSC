\documentclass[format=sigconf]{acmart}
\usepackage[utf8]{inputenc}
\usepackage[english]{babel}

%opening
\title{A Case for Telling the Truth}
\subtitle{From a Deontological Perspective}
\author{Devin Trowbridge}
\affiliation{%
  \department{Computer Science and Software Engineering}
  \institution{Auburn University }
  \email{dkt0003@auburn.edu}
  \city{Huntsville}
  \state{Alabama}
  \country{United States}}

\begin{document}

\maketitle

\section{Always telling the truth}
Truth-telling is a cornerstone of every modern system of belief. Religions declare that lying is sinful. Professional codes of ethics for healthcare, engineering, and accounting to name a few declare that lying is wrong. While ``little white lies'' are generally acceptable, lying with serious consequence goes against our societal and cultural norms. The repercussions and consequences of lying are present in mankind's songs, stories, folklore, and art. Truth-telling is so ingrained in mankind's inherent moral compass, any argument to the contrary is seemingly alien. Because honesty is so pervasive throughout every man made system of belief, truth-telling is an absolute necessity.

Finding references to the immorality of lying is incredibly easy. The fourth of the Five Precepts of Buddhism is, ``I undertake the Precept to refrain from lying, slandering...''. The second principle of medical ethics published by the American Medical Association states: ``A physician shall...be honest in all professional interactions...'' \cite{amaethics} In the the 1986 song ``You Give Love a Bad Name'', Bon Jovi sings, ``An angel's smile is what you sell / you promise me heaven, then put me through hell.'' \cite{bonjovi} These few quotes represent the smallest subset of warnings against lying in the ocean of potential source material. The dissimilarity between these quotes' sources is no mistake either. The necessity of truth-telling is present across all mediums, cultures, and organizations.

It appears as though man will continue to regard honesty as the best course of action for the foreseeable future. Certainly, people will always attempt to rationalize away lies they tell. Arguments to the contrary may reference little white lies, or even needing more context before making a determination about the morality of telling a particular lie.  However, a simple yes or not question is enough to get to the heart of the issue. If you ask any average person the point blank question, ``Is telling the truth necessary?'', they will always respond in positive affirmation. 

\medskip

\bibliographystyle{ACM-Reference-Format}
\bibliography{Module01}

\end{document}
