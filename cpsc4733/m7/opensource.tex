\documentclass[format=sigconf]{acmart}
\usepackage[utf8]{inputenc}
\usepackage[english]{babel}

%opening
\title{Open Source Ethics}
\author{Devin Trowbridge}
\affiliation{%
  \department{Computer Science and Software Engineering}
  \institution{Auburn University}
  \email{dkt0003@auburn.edu}
  \city{Huntsville}
  \state{Alabama}
  \country{United States}
}

\begin{document}

\maketitle

\section{Network Time Protocol}

Network Time Protocol is one of the oldest internet protocols in use. Sitting in the application layer of the OSI model, NTP is used for synchronizing time between computer systems over networks, most notably the Internet. Truly a foundational piece of Internet software. NTP's longevity has offered it a substantial market share, ``as the preeminent time synchronization system for Macs, Windows, and Linux computers and most servers on networks.'' \cite{ntp} Perhaps most interestingly, NTP is, ``the open source code movement's first and biggest success [story]''. \cite{ntp} Who maintains this critical piece of the Internet infrastructure? In theory its the Network Time Foundation, but in practice NTP's fate is intertwined with Harlan Stenn's. Over the years, the number of NTP maintainers has shrunk down to just Mr. Stenn who had to consider abandoning his work on NTP in 2015 due to personal finance issues. How could a maintainer for a such critical and prolific piece of infrastructure be on the verge of going broke?

The Internet software ecosystem is filled with these so-called ``'Load Bearing Internet People',...[people] who maintain the software for a critical Internet service or library, and has to do it without organizational support or a budget backing him up.`` \cite{lbip} Finding stories of LBIPs struggling financially is not uncommon either. While creators and maintainers are creating and maintaining, large money making organizations around the world are profiting using the very same open source tools offering little in return to the open source community. Of course there are exceptions, particularly in the tech industry, but many organizations have little awareness of the plight of the open source developer. If these organizations profit so much off of open source software, surely they should shoulder the cost of developing it. Therefore, organizations that profit off of open source software have an ethical obligation to fund open source software.

Companies profit the most off of open source software. \cite{gartner} 

Open source works form a critical component of our modern infrastructures. NTP, OpenSSL, Linux. \cite{ford}

Developers of open source works of software are not adequately compensated. 

Conclusion

\medskip

\bibliographystyle{ACM-Reference-Format}
\bibliography{opensource}

\end{document}
